\documentclass[12pt]{article}

\usepackage[utf8]{inputenc}
\usepackage[spanish]{babel}
\usepackage[backend=biber]{biblatex}
\usepackage{csquotes}
\usepackage{graphicx} % Inclusión de imágenes
\usepackage[left=3cm, right=3cm]{geometry} % Márgenes e interlineado
\linespread{1.5}

\bibliography{Otros cúbits superconductores}

\begin{document}

\begin{titlepage}
    \centering
    \includegraphics[width=0.7\textwidth]{logo_unir.png}\par\vspace{1cm}
    \vspace{1cm}
    {\scshape\Large Máster en Computación Cuántica\par}    
    \vspace{1cm}
    {\scshape\Large Implementación Física de un Procesador Cuántico\par}
    \vspace{3cm}
    {\Huge\bfseries Actividad 1: Otros Superconductores\par}
    \vspace{1.5cm}
    {\Large\itshape Iker Etxebarria Oseguera\par}
    \vfill
    {\large \today\par}
\end{titlepage}

\newpage

\section{Introducción}

La computación cuántica se presenta como una tecnología disruptiva que promete cambiar la forma en que procesamos información, gracias a su capacidad para realizar cálculos de manera exponencialmente más rápida que los sistemas clásicos. La implementación de procesadores cuánticos que permitan aprovechar las ventajas que ofrece la computación cuántica representa un desafío tecnológico significativo. Los procesadores cuánticos, debido a las peculiaridades del hardware cuántico, requieren un ambiente de trabajo controlado y altamente aislado para mantener la coherencia de los qubits, lo que se traduce en una necesidad de infraestructura muy específica y costosa. Además, la programación y el control de los procesadores cuánticos son complejos debido a la necesidad de trabajar con algoritmos cuánticos y la interpretación de los resultados de las mediciones cuánticas.

Estos desafíos han llevado a un impulso investigativo en diferentes áreas, como la física, la ingeniería y la informática, en busca de soluciones innovadoras que permitan superar estas dificultades. La investigación en qubits superconductores ha avanzado considerablemente en los últimos años, y actualmente se han desarrollado diferentes tecnologías de qubits superconductores, como los qubits de carga, flujo y fase, que están siendo evaluados para su implementación en procesadores cuánticos de próxima generación.

En particular, los cúbits superconductores se presentan como una de las tecnologías más prometedoras para la implementación de procesadores cuánticos. A diferencia de otros tipos de cúbits, los cúbits superconductores basados en carga, flujo y fase permiten una implementación más escalable y, por lo tanto, son objeto de investigación activa por parte de diferentes empresas e instituciones académicas.

En este trabajo se realiza una comparativa de los diferentes tipos de cúbits superconductores utilizados actualmente para la implementación de procesadores cuánticos. En particular, se estudian los cúbits de carga, flujo y fase, y se comparan sus características, ventajas e inconvenientes. Además, se analizan las empresas líderes en la implementación de procesadores cuánticos y se explican las razones por las que cada una de ellas se decanta por una tecnología específica.

\section{Fundamentos teóricos de la superconductividad}

La superconductividad es un fenómeno cuántico que se produce en algunos materiales a muy bajas temperaturas, en el que los electrones que los forman se comportan de manera coordinada y sin resistencia eléctrica. Este fenómeno se basa en la existencia de una interacción atractiva y débil \cite{bardeen_theory_1957} entre los electrones de una red cristalina, lo que da lugar a la formación de pares de Cooper, que son los responsables del fenómeno de la superconductividad. Los pares de Cooper son dos electrones que se atraen mutuamente debido a la interacción con los fonones \cite{ferrell_knight_1959}. Estos pares tienen una energía menor que los electrones individuales, lo que significa que pueden moverse a través del cristal con menos resistencia. Es decir, los pares de Cooper pueden moverse sin disipar energía debido a las colisiones con los átomos de la red cristalina \cite{bardeen_microscopic_1957} al contrario que los electrones individuales no pueden moverse a través del cristal con libertad, ya que cualquier intento de hacerlo requeriría que se rompieran los pares de Cooper.

Este fenómeno es esencial para el funcionamiento de los cúbits superconductores, ya que la ausencia de resistencia eléctrica permite que las corrientes circulen sin pérdidas y los estados cuánticos puedan mantenerse durante largos períodos de tiempo. Además, la superconductividad también permite la creación de campos magnéticos muy intensos \cite{tyler_measurements_2004}, lo que se utiliza en algunos tipos de cúbits para acoplarlos y controlar su evolución cuántica (cita).

La teoría de la superconductividad se basa en el modelo BCS propuesto por John Bardeen, Leon Cooper y Robert Schrieffer en 1957 \cite{bardeen_theory_1957}, que describe la formación de los pares de Cooper y cómo éstos se comportan en el material superconductor. Desde entonces, se han descubierto muchos materiales que presentan superconductividad a diferentes temperaturas, lo que ha permitido la construcción de cúbits superconductores basados en diferentes fenómenos cuánticos, como la carga, el flujo o la fase.

\section{Tipos de cúbits superconductores}

[Texto de tipos de cúbits superconductores]

\section{Empresas que utilizan cúbits superconductores}

[Texto de empresas que utilizan cúbits superconductores]

\section{Comparación de los diferentes tipos de cúbits superconductores}

[Texto de comparación de los diferentes tipos de cúbits superconductores]

\section{Conclusiones}

[Texto de conclusiones]


\cite{barends_superconducting_2014}
\printbibliography



\end{document}
