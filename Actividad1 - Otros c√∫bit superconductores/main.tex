\documentclass[12pt]{article}

\usepackage[utf8]{inputenc}
\usepackage[spanish]{babel}
\usepackage[backend=biber]{biblatex}
\usepackage{csquotes}
\usepackage{graphicx} % Inclusión de imágenes
\usepackage[left=3cm, right=3cm]{geometry} % Márgenes e interlineado
\linespread{1.5}

\bibliography{Otros cúbits superconductores}

\begin{document}

\begin{titlepage}
    \centering
    \includegraphics[width=0.7\textwidth]{logo_unir.png}\par\vspace{1cm}
    \vspace{1cm}
    {\scshape\Large Máster en Computación Cuántica\par}    
    \vspace{1cm}
    {\scshape\Large Implementación Física de un Procesador Cuántico\par}
    \vspace{3cm}
    {\Huge\bfseries Actividad 1: Otros Superconductores\par}
    \vspace{1.5cm}
    {\Large\itshape Iker Etxebarria Oseguera\par}
    \vfill
    {\large \today\par}
\end{titlepage}

\newpage

\section{Introducción}

\section{Fundamentos teóricos}

[Texto de fundamentos teóricos]

\section{Tipos de cúbits superconductores}

[Texto de tipos de cúbits superconductores]

\section{Empresas que utilizan cúbits superconductores}

[Texto de empresas que utilizan cúbits superconductores]

\section{Comparación de los diferentes tipos de cúbits superconductores}

[Texto de comparación de los diferentes tipos de cúbits superconductores]

\section{Conclusiones}

[Texto de conclusiones]


\cite{barends_superconducting_2014}
\printbibliography



\end{document}
