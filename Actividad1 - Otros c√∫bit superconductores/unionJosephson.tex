
\subsection{Unión Josephson}

La unión Josephson es un dispositivo compuesto por dos superconductores separados por una capa delgada de material no superconductor, conocida como barrera Josephson. En esta estructura, los pares de Cooper de los superconductores pueden atravesar la barrera gracias al efecto túnel cuántico (cita). La propiedad más importante de la unión Josephson en los cúbits es que cuando se aplica un voltaje entre los electrodos de la unión Josephson, se produce un flujo de corriente que oscila de manera periódica en el tiempo, lo que permite la generación de estados superpuestos y la realización de operaciones cuánticas lo que se denomina como efecto Josephson (cita).  \\
\begin{figure}[ht]
    \centering
    \includegraphics[width=0.5\textwidth]{Josephson Juntion.png}
    \caption{Unión Josephson y efecto túnel \citep{noauthor_squid_nodate}}
    \label{fig:Unión Josephsone}
  \end{figure}